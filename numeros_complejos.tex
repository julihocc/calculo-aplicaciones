\section{Introducción}

\subsection*{Motivación}
En estas notas, denotaremos por $\R$ el conjunto de números reales. En esta sección, procederemos de manera informal,
para motivar la definición de un número complejo y formalizar sus propiedades, en secciones posteriores. 


Supongamos que $a,b,c\in \R,$ y queremos resolver
la ecuación
$$
ax^2+bx+c=0.
$$

De manera algebraíca encontramos que las soluciones estan dadas por la fórmula
$$
x=\dfrac{-b\pm \sqrt{D}}{2a}, \, D=b^2-4ac.
$$

Si $D \geq 0,$ entonces $D$ es un número real. Sin embargo, ¿qué sucede si $D<0$?. Por la \emph{ley de los signos} si
$x,y\geq 0,$ entonces $xy\geq 0.$ De la misma manera, si $x,y<0,$ entonces $xy>0.$ En particular, para cualquier
$x\in \R,$ tenemos que $x^{2}=x\cdot x\geq 0.$ Por lo tanto, $\sqrt{D} \notin \R$ si $D<0.$

Una solución a este problema es definir el número $i=\sqrt{-1}.$ En este caso, si $D<0,$ entonces usando leyes de los
exponentes tenemos que
$$
\sqrt{D}=\sqrt{(-1)(-D)}=\sqrt{-1}\sqrt{-D}=\sqrt{-D}i.
$$
En este caso, como $D<0,$ entonces $-D>0$ y $\sqrt{-D}\in \R.$

\begin{problema}
 Las soluciones de la ecuación $x^2+1=0$ son $x=0+i1$ y $x=0+i(-1),$ o simplemente, $x=\pm i.$
\end{problema}

\begin{problema}
 Encuentre las soluciones de la siguientes ecuaciones:
 \begin{enumerate}
  \item $x^{4}+16=0,$
  \item $x^{2}-2x+2=0.$
 \end{enumerate}

\end{problema}


 Entonces, diremos que un número complejo es una cantidad de la forma
 $$
z=x+iy, \, x,y\in \R, \, i=\sqrt{-1}.
 $$
 Observe que si $x\in \R,$ podemos identificarlo con $x+i0.$

Definimos la suma de números complejos $z=x+iy,z'=x+iy'$ de la siguiente manera:
$$
z+z'=(x+x')+i(y+y').
$$

\begin{problema}
Demuestre que 
\begin{enumerate}
 \item $(x+iy)+(x'+iy')=(x'+iy')+(x+iy).$
 \item $\left[ (x+iy)+(x'+iy') \right] +(x''+iy'')= (x+iy)+\left[ (x'+iy') +(x''+iy'') \right]$
 \item $0+(x+iy)=x+iy$
 \item $(x+iy)+((-x)+i(-y))=0$
\end{enumerate}\end{problema}

Diremos que $0=0+i0$ es el \emph{neutro aditivo} en los número complejos y que $-z:=-x-iy$ es el \emph{inverso aditivo}
de $z=x+iy.$

Ahora queremos definir la multiplicación $(x+iy)(x'+iy').$ Sigamos las reglas algebraicas usuales para números reales,
salvo por la identidad $i^2=-1.$

\begin{align*}
(x+iy)(x'+iy')&= x(x'+iy')+iy(x'+iy')\\
&= xx'+x(iy')+(iy)x'+(iy)(iy') \\
&= xx' + ixy +iyx' + i^{2}yy' \\
&= (xx'-yy')+i(xy'+yx').
\end{align*}

En resumen,
 $$
zz'= (xx'-yy')+i(xy'+yx') \in \C.
 $$

\begin{problema}
Demostrar las siguientes propiedades de la multiplicación de número complejos
\begin{enumerate}
 \item $(x+iy)(x'+iy')=(x'+iy')(x+iy).$
 \item $\left[ (x+iy)(x'+iy') \right] (x''+iy'')= (x+iy)\left[ (x'+iy') (x''+iy'') \right]$
 \item $(1+i0)(x+iy)=x+iy$
 \item $(x+iy)(x-iy)=x^2+y^2.$
 \item $(x+iy)\left( \dfrac{x-iy}{x^2+y^2} \right)=1.$
\end{enumerate}
\end{problema}

Diremos que $1=1+i0$ es el \emph{neutro multiplicativo} en los número complejos y que $$
z^{-1}:=\dfrac{x-iy}{x^2+y^2} 
$$ es el \emph{inverso multiplicativo} de $z=x+iy.$

Si definimos $\bar{z}=x-iy,$ para $z=x+iy,$ podemos reescribir $$z^{-1}=\dfrac{\bar{z}}{z\bar{z}}.$$
Diremos que $\bar{z}$ es el \emph{conjugado} de $z.$

\begin{rem}
Los número reales se pueden identificar con una línea recta. Como $i$ no se puede identificar con un número en la línea
recta, se decía que este número era \emph{imaginario.} Sin embargo, podemos visualizar los números complejos (es decir,
¡dibujarlos!), para lo cuál necesitaremos ``más espacio''. Como requerimos dos números reales para describir un
complejo, tendremos que dibujarlos en el plano.
\end{rem}

\begin{problema}
\label{exe:1.1.1}
 Encuentre el resultado de las siguientes operaciones:
 \begin{enumerate}
  \item $\left( 1+i\sqrt{3} \right)\left( -1 +i\sqrt{3} \right)$
  \item $\dfrac{\frac{1}{\sqrt{2}}+i\frac{1}{\sqrt{2}}}{\sqrt{3}+i1}$
  \item $\left( \sqrt{2}+i\sqrt{6} \right)^{3}$
 \end{enumerate}

\end{problema}
